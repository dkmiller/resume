\documentclass[11pt,a4paper]{moderncv}
\moderncvtheme{classic}
\moderncvcolor{blue}

\usepackage[scale=0.75]{geometry}

\firstname{Daniel}
\familyname{Miller}
\address{120 Malott Hall, Cornell University}{Ithaca, NY 14853}
\mobile{(402)~512-4475}
\email{daniel.keegan.miller@gmail.com}
\social[github]{dkmiller}
\social[linkedin]{daniel-miller}
\homepage{www.math.cornell.edu/\string~dkmiller}

\begin{document}
\makecvtitle





\section{Research interests}

I study the statistics of Frobenius elements coming from curves of genus one or 
two. I am especially interested in the conjectured distribution of these 
Frobenius elements, rate of convergence to the conjectured Sato--Tate 
distribution, and the interactions of these with the analytic behavior of 
``curious $L$-functions'' coming from almost-everywhere continuous functions 
on the space of conjugacy classes of the associated Sato--Tate group. 





\section{Education}

\cventry{2012--2017}{Ph.D.~candidate in Mathematics}{Cornell University}{}{}{}
\cventry{2015--2017}{M.S.~in Computer Science}{Cornell University}{}{}{}

\cventry{2009--2012}{Bachelor of Science}{University of Nebraska at Omaha}{}{4.0 GPA}{}
\cvlistitem{Highest honors in Mathematics}
\cvlistitem{Minor in Computer Science}





\nocite{*}
\bibliographystyle{plain}
\bibliography{CV-sources}





\section{Conference talks}

\cvline
  {April 2013}
  {\emph{Modular curves of infinite level [after Jared Weinstein]},
  Upstate New York Number Theory Conference,
  Binghamton University}
  
\cvline
  {May 2013}
  {\emph{Perfectoid spaces},
  Ramification and Hopf--Galois module theory,
  University of Nebraska at Omaha}
  
\cvline
  {April 2014}
  {\emph{Average ranks of Selmer groups and maximal isotropic subspaces [after Bjorn Poonen]},
  Upstate New York Number Theory Conference,
  University at Buffalo}
  
\section{Other talks}

\cvline
  {November 2012}
  {\emph{Algebraic topology in positive characteristic},
  Olivetti Club,
  Cornell University}
  
\cvline
  {January 2013}
  {\emph{Taniyama-Shimura, the $R=\mathbf{T}$ theorem and Fermat-Wiles},
  Number Theory Seminar,
  Cornell University}
  
\cvline
  {April 2013}
  {\emph{Towards perfectoid spaces},
  Number Theory Seminar,
  Cornell University}
  
\cvline
  {April 2013}
  {\emph{A bestiary of Frobenii},
  Olivetti Club,
  Cornell University}

\cvline
  {July 2013}
  {\emph{Sheaves and forcing},
  Informal logic seminar,
  Cornell University}
  
\cvline
  {September 2013}
  {\emph{The Weil Conjectures for dummies},
  Number Theory Seminar,
  Cornell University}
  
\cvline
  {November 2013}
  {\emph{Taniyama-Shimura revisited},
  Number Theory Seminar,
  Cornell University}
  
\cvline
  {January 2014}
  {\emph{$L$-functions and equidistribution in number theory},
  Olivetti Club,
  Cornell University}

\cvline
  {February 2014}
  {\emph{Perfectoid spaces I: history and motivation},
  Number Theory Seminar,
  Cornell University}

\cvline
  {March 2014}
  {\emph{Perfectoid spaces II: recent applications},
  Number Theory Seminar,
  Cornell University}
  
\cvline
  {August 2014}
  {\emph{Automorphic representations and deformation theory in arithmetic geometry},
  Informal seminar on knots and primes,
  Cornell University}
  
\cvline
  {September 2014}
  {\emph{A brief tour of Grothendieck-Teichm\"uller theory},
  Olivetti Club,
  Cornell University}
  
\cvline
  {November 2014}
  {\emph{Local Langlands for $\mathrm{GL}(n)$ over $p$-adic fields [after Peter Scholze]},
  Number Theory Seminar,
  Cornell University}
  
\cvline
  {February 2015}
  {\emph{($p$-adic) Hodge theory and period rings},
  Olivetti Club,
  Cornell University}
  
\cvline
  {September 2015}
  {\emph{Torsion in the cohomology of arithmetic groups},
  Olivetti Club,
  Cornell University}





\section{Teaching experience}

\cvline{Fall 2013}{Teaching assistant for MATH 1220: Honors Calculus II}
\cvline{Spring 2014}{Teaching assistant for MATH 2220: Multivariable Calculus}
\cvline{Fall 2014}{Teaching assistant for MATH 1220: Honors Calculus II}
\cvline{Spring 2015}{Grader for MATH 6320: Graduate Algebra II}
\cvline{Fall 2015}{Grader for MATH 6310: Graduate Algebra I}
\cvline{Spring 2016}{Czar's assistant for MATH 1110/1120: Calculus I/II}





\section{Graduate coursework}

\cvlistdoubleitem{Algebraic Number Theory}{Non-Archimedean Geometry}
\cvlistdoubleitem{Algebra I \& II}{Operating Systems}
\cvlistdoubleitem{Algebraic Geometry}{Toric Varieties}
\cvlistdoubleitem{Algebraic Topology I}{Perverse Sheaves}
\cvlistdoubleitem{Arithmetic of Curves}{Ranks of Elliptic Curves}
\cvlistdoubleitem{Automorphic Forms}{Real Analysis}
\cvlistdoubleitem{Cloud Computing}{Smooth Manifolds}
\cvlistdoubleitem{Commutative Algebra}{Homological Algebra}
\cvlistdoubleitem{Lie Algebras}{Linear Algebraic Groups}





\section{Relevant skills}

\cvline{Languages}{English, (mathematical) French}
\cvline{Programming}{C, C\#, Java, \TeX, Python}





\section{Reference}

\cvline{}{\textbf{Ravi Ramakrishna}, Professor of Mathematics, Cornell University, (607)~257~6972, \href{mailto:ravi@math.cornell.edu}{\nolinkurl{ravi@math.cornell.edu}}}





\end{document}
