%!TEX TS-program = xelatex
%!TEX encoding = UTF-8 Unicode

% This template has been downloaded from:
% https://github.com/posquit0/Awesome-CV

\documentclass[11pt, letterpaper]{awesome-cv}
\geometry{left=1.4cm, top=.8cm, right=1.4cm, bottom=1.8cm, footskip=.5cm}
\fontdir[fonts/]

% Color for highlights
% Awesome Colors: awesome-emerald, awesome-skyblue, awesome-red, awesome-pink, awesome-orange
%                 awesome-nephritis, awesome-concrete, awesome-darknight
\colorlet{awesome}{awesome-emerald}
% Uncomment if you would like to specify your own color
% \definecolor{awesome}{HTML}{CA63A8}

% Set false if you don't want to highlight section with awesome color
\setbool{acvSectionColorHighlight}{true}

% If you would like to change the social information separator from a pipe (|) to something else
\renewcommand{\acvHeaderSocialSep}{\quad\textbar\quad}

\name{Daniel K.}{Miller}
\position{Mathematician{\enskip---\enskip}Data scientist}
\address{190 Pleasant Grove Road Apartment I5, Ithaca, NY 14850-2610, USA}
\mobile{(+1) 402-512-4475} 
\email{dm635@cornell.edu}
\homepage{www.math.cornell.edu/\textasciitilde dkmiller/}
\github{dkmiller}
\linkedin{dkmiller}

%\quote{``The world we live in is vastly different from the world we think we live in.'' ---Nassim Taleb}

\begin{document}

% Print the header with above personal informations
\makecvheader

% Print the footer with 3 arguments(<left>, <center>, <right>)
% Leave any of these blank if they are not needed
\makecvfooter
  {\today}
  {Daniel K.~Miller~~~·~~~Résumé}
  {\thepage}



\cvsection{Summary}
\begin{cvparagraph}

Lorem ipsum dolor sit amet, consectetur adipiscing elit. Duis ullamcorper neque sit amet lectus facilisis sed luctus nisl iaculis. Vivamus at neque arcu, sed tempor quam. Curabitur pharetra tincidunt tincidunt. Morbi volutpat feugiat mauris, quis tempor neque vehicula volutpat. Duis tristique justo vel massa fermentum accumsan. Mauris ante elit, feugiat vestibulum tempor eget, eleifend ac ipsum. Donec scelerisque lobortis ipsum eu vestibulum. Pellentesque vel massa at felis accumsan rhoncus.
\end{cvparagraph}





\cvsection{Education}

\begin{cventries}

\cventry
	{Ph.D.~in Mathematics}
	{Cornell University}
	{Ithaca, NY}
	{August 2012--PRESENT}
	{
		\begin{cvitems}
			\item{Thesis on the interplay}
		\end{cvitems}
	}
	
\cventry
	{Master's in Computer Science}
	{Cornell University}
	{Ithaca, NY}
	{August 2015--PRESENT}
	{
		\begin{cvitems}
			\item{Focus on Cloud Computing.}
		\end{cvitems}
	}
	
\cventry
	{B.S.~in Mathematics}
	{University of Nebraska Omaha}
	{Omaha, NE}
	{August 2009--July 2012}
	{
		\begin{cvitems}
			\item{Minor in Computer Science.}
			\item{Graduated \emph{summa cum laude}, with Highest Honors in Mathematics.}
		\end{cvitems}
	}
	
\end{cventries}





\cvsection{Publications}
\begin{cvparagraph}

Casey Kelleher, Daniel Miller, Trenton Osborn, and Anthony Weston. Strongly non-embeddable metric spaces. \emph{Topology Appl.}, 159(3):749--755, 2012. 

Casey Kelleher, Daniel Miller, Trenton Osborn, and Anthony Weston. Polygonal equalities and virtual degeneracy in \emph{L\textsubscript{p}}-spaces. \emph{J. Math. Anal. Appl.}, 415(1):247--268, 2014. 
\end{cvparagraph}





\cvsection{Work experience}

\begin{cventries}

\cventry
	{Cornell University}
	{Teaching Assistant}
	{Ithaca, NY}
	{August 2012--PRESENT}
	{
		\begin{cvitems}
			\item{Czar's Assistant for Calculus I and II.}
			\item{Teaching assistant for Honors Calculus II and Multivariable Calculus.}
			\item{Grader for Graduate Algebra I and II.}
		\end{cvitems}
	}

\cventry
	{Chesterton House}
	{Resident Advisor}
	{Ithaca, NY}
	{August 2013--May 2014}
	{}

\cventry
	{University of Nebraska Omaha}
	{Tutor at the Math--Science Learning Center}
	{Omaha, NE}
	{d}
	{}
	
\end{cventries}





\cvsection{Relevant coursework and skills}
\begin{cvparagraph}

\textbf{Coursework:} Algebraic topology --- Cloud computing --- Database management systems --- Data structures --- Functional programming --- Linear algebraic groups --- Object-oriented programming --- Operating systems --- Probability theory --- Real analysis --- Smooth manifolds

\textbf{Programming:} C --- C\# --- Java --- \TeX --- Python
\end{cvparagraph}


\end{document}
