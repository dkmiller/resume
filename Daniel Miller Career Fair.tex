%!TEX TS-program = xelatex

\documentclass[11pt, letterpaper]{awesome-cv}
\geometry{left=1.4cm, top=.8cm, right=1.4cm, bottom=1.8cm, footskip=.5cm}
\fontdir[fonts/]

% Color for highlights
\colorlet{awesome}{awesome-emerald}

\name{}{DANIEL MILLER}
\mobile{402.512.4475} 
\email{dm635@cornell.edu}
\linkedin{daniel-miller}

\begin{document}
\makecvheader





\cvsection{Summary}

\begin{cvparagraph}
Math Ph.D.~candidate enthusiastic about applying mathematical and computational background to understanding real-world data sets in a fast-paced environment. 
Have experience developing algorithms for understanding the statistics of large data sets coming from number theory, and led the creation of a scalable, distributed, cloud-hosted website. 
Interested in developing new methods for getting the most from existing large data sets.
\end{cvparagraph}





\cvsection{Education}

\begin{cventries}

\cventry
	{Ph.D.~in Mathematics}
	{Cornell University}
	{}
	{August 2012--May 2017}
	{
		\begin{cvitems}
			\item{Coordinated logistics, teaching, and grading for a course with 300 students and 12 faculty.}
			\item{Assisted in teaching mathematics at undergraduate and graduate levels.}
			\item{Won the Eleanor Norton York Award for excellent collaboration and rapid research progress.}
		\end{cvitems}
	}
	
\cventry
	{Master's in Computer Science}
	{Cornell University}
	{}
	{August 2015--May 2017}
	{
		\begin{cvitems}
			\item{Managed the creation of a location-centric auction site written in C\# and hosted on Azure.}
			\item{Tested the site for scalability to 2K requests per second.}
			\item{Collaborated in writing a CPU scheduler and gossip-based networking protocol in C.}
		\end{cvitems}
	}
	
\cventry
	{B.S.~in Mathematics}
	{University of Nebraska Omaha}
	{}
	{August 2009--August 2012}
	{
		\begin{cvitems}
			\item{Minored in Computer Science, graduated \emph{summa cum laude}, GPA 4.0.}
			\item{Dean's List all semesters, Highest Honors in Mathematics, with senior thesis.}
			\item{Designed and wrote a compiler for a C-style programming language in Standard ML.}
		\end{cvitems}
	}
	
\end{cventries}





\cvsection{Research Experience}

\begin{cventries}

\cventry
	{Cornell University}
	{Ph.D.~Research}
	{}
	{August 2013--May 2017}
	{
		\begin{cvitems}
			\item{Developing and implementing new techniques for computing \emph{G}-star discrepancy, used in numerical integration.}
			\item{Created sample sequences to disprove a conjecture on the discrepancy of data coming from elliptic curves.}
			\item{Proved precise connections between discrepancy of a sequence and analytic properties of an associated \emph{L}-function.}
			\item{Demonstrated non-scalability of an algorithm for computing torsion in the cohomology of locally symmetric spaces.}
		\end{cvitems}
	}
		
\cventry
	{University of Arizona}
	{Arizona Winter School}
	{}
	{May 2014, May 2016}
	{
		\begin{cvitems}
			\item{Wrote scalable code to test a new version of the Lang--Trotter conjecture to high precision.}
			\item{Formulated and proved a generalized version of the Bertini Smoothness Theorem.}
		\end{cvitems}
	}
	
\cventry
	{Cornell University}
	{Summer Mathematics Institute}
	{}
	{Summer 2011}
	{
		\begin{cvitems}
			\item{Collaborated to created a high-dimensional example that disproved a conjecture.}
			\item{Published \href{http://dx.doi.org/10.1016/j.topol.2011.11.041}{\emph{Strongly non-embeddable metric spaces}}. Topology Appl.~\textbf{159} (2012), no.3, 749--755.}
			\item{Published \href{http://dx.doi.org/10.1016/j.jmaa.2014.01.063}{\emph{Polygonal equalities and virtual degeneracy in L\textsubscript{p} spaces}}. J.~Math.~Anal.~Appl.~\textbf{415} (2014), no.1, 247--268.}
		\end{cvitems}
	}
\end{cventries}





\cvsection{Skills}

\begin{cvparagraph}
\textbf{Programming:} C\#, Python, Java, ASP.NET, C, Sage, and \LaTeX.

\textbf{Web:} \href{https://azure.microsoft.com/}{Azure}, \href{https://aws.amazon.com/}{Amazon Web Services}, HTML, and Google's \href{https://getmdl.io/}{Material Design Lite}.
\end{cvparagraph}





\end{document}
