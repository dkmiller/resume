%!TEX TS-program = xelatex

\documentclass[11pt, letterpaper]{awesome-cv}
\geometry{
	left     = 1.4cm,
	top      = 1.4cm,
	right    = 1.4cm,
	bottom   = 1.4cm,
	footskip = 0.5cm
}
\fontdir[fonts/]

% Color for highlights
\colorlet{awesome}{black}

\name{}{DANIEL MILLER}
\mobile{(+1) 402.512.4475} 
\email{daniel.keegan.miller@gmail.com}
\homepage{www.math.cornell.edu/\textasciitilde dkmiller}

\begin{document}
\makecvheader





\cvsection{Research interests}

\begin{cvparagraph}
Statistics on compact Lie groups and its connections to the analytic properties 
of Dirichlet series. The recently-proved Sato--Tate Conjecture predicts the 
distribution of the set of Frobenius eigenvalues of an elliptic curve, but not 
the rate at which the set of Frobenii converges to the predicted distribution. 
I am working on providing strong computational evidence for a stronger 
conjecture, due to Akiyama--Tanigawa, that predicts this rate of convergence.
This involves creating and implementing new algorithms for computing the 
discrepancy of large sequences of points in low dimensions. 
I am also linking their conjecture to the analytic properties of Dirichlet 
series that can be constructed from arbitrary equidistributed sequences in 
compact Lie groups. This involves using Diophantine approximation to construct 
sequences with slowly decaying discrepancy, but very small exponential sums. 
\end{cvparagraph}





\cvsection{Education}

\begin{cventries}

\cventry
	{Ph.D.~in Mathematics, Cornell University}
	{}{}
	{August 2012--May 2017}
	{
		\begin{cvitems}
			\item{Dissertation topic: Discrepancy on compact Lie groups, Dirichlet series, and the strong Sato--Tate conjecture}
			\item{Adviser: Ravi Ramakrishna}
			\item{Won the Eleanor Norton York Award for excellent collaboration and rapid research progress.}
		\end{cvitems}
	}
	
\cventry
	{M.S.~in Computer Science, Cornell University}
	{}{}
	{August 2015--May 2017}
	{
		\begin{cvitems}
			\item{Managed the creation of a location-centric auction site written in C\# and hosted on Azure.}
			\item{Collaborated in writing a CPU scheduler and gossip-based networking protocol in C.}
		\end{cvitems}
	}
	
\cventry
	{B.S.~in Mathematics, University of Nebraska Omaha}
	{}{}
	{August 2009--August 2012}
	{
		\begin{cvitems}
			\item{Minored in Computer Science, graduated \emph{summa cum laude}, GPA 4.0.}
			\item{Dean's List all semesters, Highest Honors in Mathematics, with senior thesis.}
		\end{cvitems}
	}
	
\end{cventries}






\cvsection{Publications}

\begin{cvparagraph}
Casey Kelleher, Daniel Miller, Trenton Osborn, and Anthony Weston. \href{http://dx.doi.org/10.1016/j.topol.2011.11.041}{\emph{Strongly non-embeddable metric spaces}}. Topology Appl.~\textbf{159} (2012), no.3, 749--755.

Casey Kelleher, Daniel Miller, Trenton Osborn, and Anthony Weston. \href{http://dx.doi.org/10.1016/j.jmaa.2014.01.063}{\emph{Polygonal equalities and virtual degeneracy in L\textsubscript{p} spaces}}. J.~Math.~Anal.~Appl.~\textbf{415} (2014), no.1, 247--268.
\end{cvparagraph}





\cvsection{Conference talks}

\begin{cvhonors}
	\cvhonor
	{2013}
	{Modular curves of infinite level [after Jared Weinstein]}
	{SUNY Binghamton}
	
	\cvhonor
	{2013}
	{Perfectoid spaces}
	{University of Nebraska}
	
	\cvhonor
	{2014}
	{Average ranks of Selmer groups and maximal isotropic subspaces [after Bjorn Poonen]}
	{SUNY Buffalo}
\end{cvhonors}





\cvsection{Teaching experience}

\begin{cvhonors}
	\cvhonor
	{2013}
	{Teaching assistant for MATH 1220: Honors Calculus II}
	{}
	
	\cvhonor
	{2014}
	{Teaching assistant for MATH 2220: Multivariable Calculus}
	{}
	
	\cvhonor
	{2014}
	{Teaching assistant for MATH 1220: Honors Calculus II}
	{}
	
	\cvhonor
	{2015}
	{Grader for MATH 6320: Graduate Algebra II}
	{}
	
	\cvhonor
	{2015}
	{Grader for MATH 6310: Graduate Algebra I}
	{}
	
	\cvhonor
	{2016}
	{Czar's assistant for MATH 1110/1120: Calculus I/II}
	{}
	
	\cvhonor
	{2016}
	{Grader for MATH 3040: ``Prove it!''}
	{}
\end{cvhonors}





\cvsection{Other talks}

\begin{cvhonors}
	\cvhonor
	{2012}
	{Algebraic topology in positive characteristic}
	{Cornell University}
	
	\cvhonor
	{2013}
	{Taniyama-Shimura, the \emph{R} = \textbf{T} theorem and Fermat--Wiles}
	{Cornell University}
	
	\cvhonor
	{2013}
	{Towards perfectoid spaces}
	{Cornell University}
	
	\cvhonor
	{2013}
	{A bestiary of Frobenii}
	{Cornell University}
	
	\cvhonor
	{2013}
	{Sheaves and forcing}
	{Cornell University}
	
	\cvhonor
	{2013}
	{The Weil Conjectures for dummies}
	{Cornell University}
	
	\cvhonor
	{2013}
	{Taniyama--Shimura revisited}
	{Cornell University}
	
	\cvhonor
	{2014}
	{\emph{L}-functions and equidistribution in number theory}
	{Cornell University}
	
	\cvhonor
	{2014}
	{Perfectoid spaces I: history and motivation}
	{Cornell University}
	
	\cvhonor
	{2014}
	{Perfectoid spaces II: recent applications}
	{Cornell University}
	
	\cvhonor
	{2014}
	{Automorphic representations and deformation theory in arithmetic geometry}
	{Cornell University}

	\cvhonor
	{2014}
	{A brief tour of Grothendieck--Teichm\"uller theory}
	{Cornell University}
	
	\cvhonor
	{2014}
	{Local Langlands for GL(\emph{n}) over \emph{p}-adic fields [after Peter Scholze]}
	{Cornell University}
	
	\cvhonor
	{2015}
	{(\emph{p}-adic) Hodge theory and period rings}
	{Cornell University}
	
	\cvhonor
	{2015}
	{Torsion in the cohomology of arithmetic groups}
	{Cornell University}
\end{cvhonors}





\cvsection{Graduate Coursework}

\begin{cvparagraph}
Algebraic number theory, Algebra I\&II, Algebraic geometry, Algebraic topology, 
Arithmetic of curves, Automorphic forms, Cloud computing, Commutative algebra, 
Lie algebras, Non-Archimedean geometry, Toric varieties, Perverse sheaves, 
Ranks of elliptic curves, Real analysis, Smooth manifolds, Homological algebra, 
Linear algebraic groups, Principles of distributed systems.
\end{cvparagraph}





\cvsection{Relevant skills}

\begin{cvparagraph}
\textbf{Code}: Python, Sage, \LaTeX, Java, C\#, ASP.NET, and C.
\end{cvparagraph}





\cvsection{Reference}

\begin{cvparagraph}
\textbf{Ravi Ramakrishna}, Chair of Cornell University Mathematics Department, 
(+1) 607.257.6972, \href{mailto:ravi@math.cornell.edu}{ravi@math.cornell.edu}.
\end{cvparagraph}





\end{document}
