%!TEX TS-program = xelatex

\documentclass[11pt, letterpaper]{awesome-cv}
\geometry{left=1.4cm, top=.8cm, right=1.4cm, bottom=1.8cm, footskip=.5cm}
\fontdir[fonts/]

% Color for highlights
\colorlet{awesome}{black}

% Colors for text
\definecolor{darktext}{gray}{0}
\definecolor{text}{gray}{0}
\definecolor{graytext}{gray}{0}
\definecolor{lighttext}{gray}{0}

% Set false if you don't want to highlight section title
\setbool{acvSectionColorHighlight}{false}

% Social information separator.
\renewcommand{\acvHeaderSocialSep}{\quad\textbar\quad}

\name{}{Daniel Miller}
%\position{Mathematician{\enskip---\enskip}Data scientist}
\mobile{(+1) 402-512-4475} 
\email{daniel.keegan.miller@gmail.com}
\linkedin{daniel-miller}

\begin{document}
\makecvheader





\cvsection{Summary}
\begin{cvparagraph}

% Lay summary
I'm a math Ph.D.~student who works on the statistics of large data sets coming from number theory. 
\end{cvparagraph}





\cvsection{Education}

\begin{cventries}

\cventry
	{Ph.D.~in Mathematics}
	{Cornell University}
	{Ithaca, NY}
	{August 2012--May 2017}
	{
		\begin{cvitems}
			\item{Oversaw logistics and planning for a course with 300 students.}
			\item{Taught classes at undergraduate and graduate levels.}
			\item{Relevant courses: Smooth Manifolds, Algebraic Topology, Real Analysis, and Algebraic Groups.}
			\item{Won the Eleanor Norton York Award on the basis of my achievements.}
		\end{cvitems}
	}
	
\cventry
	{Master's in Computer Science}
	{Cornell University}
	{Ithaca, NY}
	{August 2015--May 2017}
	{
		\begin{cvitems}
			\item{Created a distributed, cloud-based, location-centric auction site, and tested it for scalability.}
			\item{Relevant courses: Cloud Computing, Distributed Computing, Operating Systems.}
		\end{cvitems}
	}
	
\cventry
	{B.S.~in Mathematics}
	{University of Nebraska Omaha}
	{Omaha, NE}
	{August 2009--August 2012}
	{
		\begin{cvitems}
			\item{Minor in Computer Science, graduated \emph{summa cum laude}, with Highest Honors in Mathematics.}
			\item{Relevant courses: Databases, Data structures, Functional programming, Probability.}
		\end{cvitems}
	}
	
\end{cventries}





\cvsection{Research Experience}

\begin{cventries}

\cventry
	{Cornell University (PhD)}
	{Computational statistics of elliptic curves}
	{}
	{August 2015--May 2017}
	{
		\begin{cvitems}
			\item{I am developing new techniques for computing the \emph{G}-star discrepancy of large sequences. Also, I have proved precise connections between the discrepancy of a sequence and the analytic properties of an associated \emph{L}-function. Finally, I have streamlined the traditional foundations of Galois deformation theory.}
		\end{cvitems}
	}
	
\cventry
	{Cornell University (Undergraduate)}
	{Summer Mathematics Institute}
	{}
	{Summer 2011}
	{
		\begin{cvitems}
			\item{With A.~Weston, C.~Kelleher, and T.~Osborn, created a complex new example that disproved a long-standing conjecture.}
			\item{\emph{Strongly non-embeddable metric spaces}. Topology Appl.~\textbf{159} (2012), no.3, 749--755.}
			\item{\emph{Polygonal equalities and virtual degeneracy in L\textsubscript{p} spaces}. J.~Math.~Anal.~Appl.~\textbf{415} (2014), no.1, 247--268.}
		\end{cvitems}
	}
	
\cventry
	{University of Arizona}
	{Arizona Winter School}
	{}
	{May 2016}
	{
		\begin{cvitems}
			\item{Formulated a version of the Lang--Trotter conjecture for a new class of objects and provided strong numerical evidence.}
		\end{cvitems}
	}

\cventry
	{University of Nebraska Omaha}
	{Fund for Undergraduate Scholarly Experiences}
	{}
	{Summer 2012}
	{
		\begin{cvitems}
			\item{Created a simpler and more robust approach to a recent and important theorem in Hopf--Galois theory.}
		\end{cvitems}
	}
	
\end{cventries}





\cvsection{Skills and Activities}

\begin{cvparagraph}

\textbf{Programming:} C\#, Java, Python, C, and \TeX.
\end{cvparagraph}

\begin{cventries}
\cventry
	{Chesterton House}
	{Resident Assistant}
	{Ithaca, NY}
	{August 2013--May 2014}
	{
		\begin{cvitems}
			\item{Coordinated events, finances, and recruiting for a living center.}
		\end{cvitems}
	}
	
\end{cventries}





\end{document}
